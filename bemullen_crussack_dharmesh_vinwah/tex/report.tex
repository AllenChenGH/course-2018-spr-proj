       \documentclass[journal, a4paper]{IEEEtran}

% some very useful LaTeX packages include:

%\usepackage{cite}      % Written by Donald Arseneau
                        % V1.6 and later of IEEEtran pre-defines the format
                        % of the cite.sty package \cite{} output to follow
                        % that of IEEE. Loading the cite package will
                        % result in citation numbers being automatically
                        % sorted and properly "ranged". i.e.,
                        % [1], [9], [2], [7], [5], [6]
                        % (without using cite.sty)
                        % will become:
                        % [1], [2], [5]--[7], [9] (using cite.sty)
                        % cite.sty's \cite will automatically add leading
                        % space, if needed. Use cite.sty's noadjust option
                        % (cite.sty V3.8 and later) if you want to turn this
                        % off. cite.sty is already installed on most LaTeX
                        % systems. The latest version can be obtained at:
                        % http://www.ctan.org/tex-archive/macros/latex/contrib/supported/cite/
 \usepackage{algorithm}
\usepackage[noend]{algpseudocode}
\makeatletter
\def\BState{\State\hskip-\ALG@thistlm}
\makeatother
\usepackage{amssymb}
\usepackage{graphicx}   % Written by David Carlisle and Sebastian Rahtz
                        % Required if you want graphics, photos, etc.
                        % graphicx.sty is already installed on most LaTeX
                        % systems. The latest version and documentation can
                        % be obtained at:
                        % http://www.ctan.org/tex-archive/macros/latex/required/graphics/
                        % Another good source of documentation is "Using
                        % Imported Graphics in LaTeX2e" by Keith Reckdahl
                        % which can be found as esplatex.ps and epslatex.pdf
                        % at: http://www.ctan.org/tex-archive/info/

%\usepackage{psfrag}    % Written by Craig Barratt, Michael C. Grant,
                        % and David Carlisle
                        % This package allows you to substitute LaTeX
                        % commands for text in imported EPS graphic files.
                        % In this way, LaTeX symbols can be placed into
                        % graphics that have been generated by other
                        % applications. You must use latex->dvips->ps2pdf
                        % workflow (not direct pdf output from pdflatex) if
                        % you wish to use this capability because it works
                        % via some PostScript tricks. Alternatively, the
                        % graphics could be processed as separate files via
                        % psfrag and dvips, then converted to PDF for
                        % inclusion in the main file which uses pdflatex.
                        % Docs are in "The PSfrag System" by Michael C. Grant
                        % and David Carlisle. There is also some information
                        % about using psfrag in "Using Imported Graphics in
                        % LaTeX2e" by Keith Reckdahl which documents the
                        % graphicx package (see above). The psfrag package
                        % and documentation can be obtained at:
                        % http://www.ctan.org/tex-archive/macros/latex/contrib/supported/psfrag/

%\usepackage{subfigure} % Written by Steven Douglas Cochran
                        % This package makes it easy to put subfigures
                        % in your figures. i.e., "figure 1a and 1b"
                        % Docs are in "Using Imported Graphics in LaTeX2e"
                        % by Keith Reckdahl which also documents the graphicx
                        % package (see above). subfigure.sty is already
                        % installed on most LaTeX systems. The latest version
                        % and documentation can be obtained at:
                        % http://www.ctan.org/tex-archive/macros/latex/contrib/supported/subfigure/

\usepackage{url}        % Written by Donald Arseneau
                        % Provides better support for handling and breaking
                        % URLs. url.sty is already installed on most LaTeX
                        % systems. The latest version can be obtained at:
                        % http://www.ctan.org/tex-archive/macros/latex/contrib/other/misc/
                        % Read the url.sty source comments for usage information.

%\usepackage{stfloats}  % Written by Sigitas Tolusis
                        % Gives LaTeX2e the ability to do double column
                        % floats at the bottom of the page as well as the top.
                        % (e.g., "\begin{figure*}[!b]" is not normally
                        % possible in LaTeX2e). This is an invasive package
                        % which rewrites many portions of the LaTeX2e output
                        % routines. It may not work with other packages that
                        % modify the LaTeX2e output routine and/or with other
                        % versions of LaTeX. The latest version and
                        % documentation can be obtained at:
                        % http://www.ctan.org/tex-archive/macros/latex/contrib/supported/sttools/
                        % Documentation is contained in the stfloats.sty
                        % comments as well as in the presfull.pdf file.
                        % Do not use the stfloats baselinefloat ability as
                        % IEEE does not allow \baselineskip to stretch.
                        % Authors submitting work to the IEEE should note
                        % that IEEE rarely uses double column equations and
                        % that authors should try to avoid such use.
                        % Do not be tempted to use the cuted.sty or
                        % midfloat.sty package (by the same author) as IEEE
                        % does not format its papers in such ways.

\usepackage{amsmath}    % From the American Mathematical Society
                        % A popular package that provides many helpful commands
                        % for dealing with mathematics. Note that the AMSmath
                        % package sets \interdisplaylinepenalty to 10000 thus
                        % preventing page breaks from occurring within multiline
                        % equations. Use:
%\interdisplaylinepenalty=2500
                        % after loading amsmath to restore such page breaks
                        % as IEEEtran.cls normally does. amsmath.sty is already
                        % installed on most LaTeX systems. The latest version
                        % and documentation can be obtained at:
                        % http://www.ctan.org/tex-archive/macros/latex/required/amslatex/math/



% Other popular packages for formatting tables and equations include:

%\usepackage{array}
% Frank Mittelbach's and David Carlisle's array.sty which improves the
% LaTeX2e array and tabular environments to provide better appearances and
% additional user controls. array.sty is already installed on most systems.
% The latest version and documentation can be obtained at:
% http://www.ctan.org/tex-archive/macros/latex/required/tools/

% V1.6 of IEEEtran contains the IEEEeqnarray family of commands that can
% be used to generate multiline equations as well as matrices, tables, etc.

% Also of notable interest:
% Scott Pakin's eqparbox package for creating (automatically sized) equal
% width boxes. Available:
% http://www.ctan.org/tex-archive/macros/latex/contrib/supported/eqparbox/

% *** Do not adjust lengths that control margins, column widths, etc. ***
% *** Do not use packages that alter fonts (such as pslatex).         ***
% There should be no need to do such things with IEEEtran.cls V1.6 and later.


% Your document starts here!
\begin{document}

% Define document title and author
    \title{Student Impact on Move-in Week the Greater Boston Area}
    \author{Brooke Mullen, Claire Russack, Dharmesh Tarapore, Vincent Wahl
    \\bemullen@bu.edu, crussack@bu.edu, dharmesh@bu.edu, vinwah@bu.edu
    }
    \markboth{Boston University: CS 591 L1}{}
    \maketitle

% Write abstract here
\begin{abstract}
    Move in week is universally chaotic and a generally unpleasant experience for both, college students and local residents \cite{painfulmovein}. Despite universities' meticulous attempts to mitigate trouble, the logistical concerns presented by the sudden influx of over 30,000 students are intractable. In this project, we attempt to identify specific factors affecting residents' quality of life (as a result of move-in week) and study ways to improve it.   
\end{abstract}

% Each section begins with a \section{title} command
\section{Introduction}
    % \PARstart{}{} creates a tall first letter for this first paragraph
    \IEEEPARstart{}Over 50,000 new and returning students move in to dormitories and apartments in the city of Boston each year \cite{moveinstats}. Known colloquially as "move-in week", this 7-day period between late August and early September wreaks considerable havoc over residents' lives. Traffic delays, overcrowded public transport, inadequate parking spaces, and the infamous \textit{"Storrowed"} trucks \cite{storrowed} are among some of the difficulties that irk both, students and locals.
\\Due to the nature of an academic calendar for most universities, the lease cycle tends to be September 1st. This rapid movement in a dense area affects the overall quality of life in the city, often for the worse. [TODO: explain why we care(answering 2 questions...)].
\\Mensurability notwithstanding, conventional proxies\footnote{The GDP (Gross Domestic Product) is often used as an approximate measure of quality of life, in conjunction with other metrics such as the HDI (Human Development Index) and Gini Coefficient.} for quality of life disregard (among other things) societal happiness: a critical adjunct to traditional economic metrics \cite{happytext}. 
To remedy this, we propose an approach that utilizes open data sourced from Boston City's \textit{Analyze Boston} data portal \cite{bosdatalink} and posts made on the microblogging platform, Twitter \cite{twitterlink}. In particular, we use the \textit{CityScore}, \textit{311 Service Requests}, and \textit{Boston Fire Incident Reporting} datasets to identify the demand for specific city resources across multiple time periods as a function of user satisfaction.
\\The rest of this paper is organized as follows: in section II, we document related research that inspired this project. In section III, we describe each of the 4 chosen datasets in greater detail to focus on their merits, potential limitations, and the methods we used to discern their relative importance in estimating the city's quality of life. Section IV charts our results and outlines their implications. Section V concludes the report with an outlook on future research.
%To quantify the quality of the city on a given day, we used the Analyze Boston's City Score dataset. The Boston City Score is an initiative created to monitor the overall health of the city on a daily basis. It is calculated by aggregating various metrics, both additive and negative, into a single score. This dataset is one that was central to our project and was used as a significant way of determining metrics had on the quality of life.
%A solution that the city poses is temporary parking and traffic restrictions in areas that usually surround areas with a high density of student movers. Due to minimal options for transportation and the high movement, renters do not have space or resources to deal with their trash and furniture, leaving them to dispose of it in their surroundings. It is not surprising to find refrigerators, mattresses, and desks littered along the sides of the streets during this time. Analyze Boston's 311 Service Requests records these violations and their locations. With the understanding that the number of violations affects the quality of life on a given day in Boston, one of our aims became to find a way to serve requests in the most timely way. To make this solution concrete but also efficient, our team worked to display the minimum number of dispatchers required to respond to service requests within 30 seconds of each inquiry. 
%Another metric that affects the quality of life in Boston is fire incidents in the city. The dataset, ‘Fire Incident Reporting’, also comes from Analyze Boston. The data set consists of incident details and location. It is non-trivial that a residential or commercial fire will negatively affect the quality of life so our question became finding the mean point of where incidents occur. 
%In the City Score dataset, one of the features quantifies the amount of public library users on a given day. This metric is one that is additive to the city score. While students could potentially be detrimental to the quality of Boston’s life, it was possible that they could also be boosting it with increased library attendance. Two different datasets were created from City Score based on the library attendance metric. The first consisted of attendance when university was in session and the second was when it was not in session. 

% Related Work
\section{Related Work}
    Jim Haddadin explored the relationship between move-in week and garbage disposal concerns in \cite{trashcity}. Noting a sharp rise in the number of code violations around university residences during move-in week, Haddadin highlights the consequences (and inevitability) of improper trash disposal [TODO: describe more]. In \cite{happytext}, Dodds et al address the subjectivity and vagueness inherent to estimating happiness among people by mining 46 billion tweets to uncover temporal variations in happiness and information levels over timescales ranging from hours to years.Their remote-sensing 'hedonometer' algorithm generated a rich source of information about short-term, experential happiness in a population and its causes. We use a similar, albeit slightly modified approach to understand residents' moods during move-in week.

\section{Approach}
Subsections of how each of us individually contributed to each part of the project
\subsection{Vincent} 
\subsection{Claire}
\subsection{Library Attendance Analysis}
The goal of analyzing library visits to city score started out with questioning whether the number of visits actually went up during university session. The two datasets, were filtered based on the "ETL\_LOAD\_DATE" feature from the City Score data set:
\newline Students In Session
\begin{itemize}
    \item September
    \item October 
    \item November
    \item December 1st-15th
    \item January 16th-31st
    \item February
    \item March
    \item April
    \item May
\end{itemize}
Students Not In Session: 
\begin{itemize}
    \item January 1st-15th
    \item July
    \item June
    \item August
    \item December 16th-31st
\end{itemize}
\subsection{Dharmesh}
\section{Results}
Results go here.
\section{Conclusion}
Conclusion goes here.
\section{Outlook}

% Now we need a bibliography:
\begin{thebibliography}{5}

    %Each item starts with a \bibitem{reference} command and the details thereafter.
    \bibitem{painfulmovein}
    Landry, Lauren. "Boston, You've Been Warned: The College Students Are Here.? Americaninno.com, 25 Aug. 2014, www.americaninno.com/boston/move-in-dates-for-boston-college-students-back-to-school-in-boston/.
    
    \bibitem{moveinstats}
    United States, Congress, Meade, Peter. "Boston by the Numbers: Colleges and Universities" Boston Redevelopment Authority. www.bostonplans.org/getattachment/1770c181-7878-47ab-892f-84baca828bf3 2011.
    
    \bibitem{storrowed}
    Slane, Kevin. "Trillium Has a New Beer Named for Trucks That Get 'Storrowed'." Boston.com, The Boston Globe, 2 Feb. 2018, https://www.boston.com/culture/lifestyle/2018/02/02/trillium-has-a-new-beer-named-for-trucks-that-get-storrowed.
    
    \bibitem{happytext}
    Dodds, Peter Sheridan, et al. "Temporal patterns of happiness and information in a global social network: Hedonometrics and Twitter." PloS one 6.12 (2011): e26752.
    
    \bibitem{bosdatalink}
    Analyze Boston, https://data.boston.gov/.

   \bibitem{twitterlink}
   Twitter, https://twitter.com.
   
   \bibitem{trashcity}
   J.~Haddidin,  \emph{Trash City: How Does Moving Week Impact the Quality of Life in Boston?},\relax Analyze Boston,  \emph{https://data.boston.gov/showcase/trash-city}
   
  


\end{thebibliography}

% Your document ends here!
\end{document}